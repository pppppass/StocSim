%! TeX encoding = UTF-8
%! TeX program = LuaLaTeX

\documentclass[english, nochinese]{pnote}
\usepackage{bm}
\usepackage[paper, cgu]{pdef}

\DeclareMathOperator\ope{\mathrm{E}}

\title{Answers to Exercises (Lecture 16)}
\author{Zhihan Li, 1600010653}
\date{December 26, 2018}

\begin{document}

\maketitle

\textbf{Problem 1.} \textit{Answer.} According to independence and stable increment, we may set $ t_n = 0 $ and $ t_{ n + 1 } = t $ and sample from
\begin{gather}
\Delta Z_1 = \int_0^t W_s \sd s,
\Delta Z_2 = \int_0^t s \sd W_s
\end{gather}
and $ \Delta W_n = W_t $. We note that the joint distribution $ \rbr{ Z_1, Z_2, W_n } $ is a Gaussian distribution. Since
\begin{equation}
\ope \rbr{ \int_0^t W_s \sd s }^2 = \int_0^t \int_0^t s \wedge r \sd r \sd s = \frac{1}{3} t^3,
\end{equation}
and
\begin{equation}
\ope \rbr{ \int_0^t W_s \sd s } W_t d= \int_0^t s \sd s = \frac{1}{2} t^2,
\end{equation}
and
\begin{equation}
\ope W_t^2 = t,
\end{equation}
we may sample $ \rbr{ Z_1, W_n } $ from $ \mathcal{N} \rbr{ 0, \bm{\Sigma} } 
$ where
\begin{equation}
\bm{\Sigma} = \msbr{ t^3 / 3 & t^2 / 2 \\ t^2 / 2 & t }.
\end{equation}
After that, $Z_2$ can be found by the relation
\begin{equation}
Z_1 + Z_2 = t W_t,
\end{equation}
or say
\begin{equation}
Z_2 = t W_n - Z_1.
\end{equation}

\textbf{Problem 2.} \textit{Proof.} We have
\begin{equation}
e_{ n + 1 } = e_n + \int_{t_n}^{t_{ n + 1 }} \rbr{ b \rbr{X_t} - b \rbr{X_n} } \sd t,
\end{equation}
while leaving out high order terms we have (we assume $b''$ is bounded in order to safely ignore $ b'' \rbr{X_s} $ terms)
\begin{equation}
\begin{split}
&\ptrel{=} \ope \rbr{ \int_{t_n}^{t_{ n + 1 }} b \rbr{X_t} - b \rbr{X_{t_n}} \sd t }^2 \\
&\le C_1 \ope \rbr{ \int_{t_n}^{t_{ n + 1 }} \int_{t_n}^t \rbr{ \rbr{ b \rbr{X_s} b' \rbr{X_s} + \frac{1}{2} b'' \rbr{X_s} } \sd s + b' \rbr{X_s} \sd W_s } \sd t }^2 \\
&\le C_2 \Delta t_n^4 + \ope \rbr{ \int_{t_n}^{t_{ n + 1 }} \int_{t_n}^t b' \rbr{X_s} \sd W_s \sd t }^2
\end{split}
\end{equation}
Since $b''$ is bounded, $b'$ is bounded (in an appropriate, finite region) and therefore
\begin{equation}
\begin{split}
&\ptrel{=} \ope \rbr{ \int_{t_n}^{t_{ n + 1 }} \int_{t_n}^t b' \rbr{X_s} \sd W_s \sd t }^2 \\
&\le \ope \rbr{ \int_{t_n}^{t_{ n + 1 }} \rbr{ t_{ n + 1 } - s } b' \rbr{X_s} \sd W_s }^2 \\
&\le M \int_{t_n}^{t_{ n + 1 }} \rbr{ t_{ n + 1 } - s }^2 \sd s = \frac{1}{3} M \rbr{ \Delta t_n }^3.
\end{split}
\end{equation}
This estimation shows
\begin{equation}
\ope \rbr{ \int_{t_n}^{t_{ n + 1 }} b \rbr{X_t} - b \rbr{X_{t_n}} \sd t }^2 \le C_4 \rbr{ \Delta t_n }^3
\end{equation}
and hence
\begin{equation}
\begin{split}
\ope e_{ n + 1 }^2 &\le \rbr{ 1 + L_1 \Delta t_n } \ope e_n^2 + \rbr{ 1 + \frac{1}{ \Delta t_n } } \ope \rbr{ \int_{t_n}^{t_{ n + 1 }} \rbr{ b \rbr{X_t} - b \rbr{X_{t_n}} } \sd t }^2 \\
&\le \rbr{ 1 + L_1 \Delta t_n } \ope e_n^2 + C_4 \rbr{ 1 + \Delta t_n } \rbr{ \Delta t_n }^2.
\end{split}
\end{equation}
Applying Gronwall’s inequality yields
\begin{equation}
\ope e_n^2 \le C_5 \rbr{ \Delta t_n }^2,
\end{equation}
i.e. first order convergence.
\hfill$\Box$

\textbf{Problem 3.} \textit{Proof.} We have
\begin{equation}
\sd X_t^{ 2 r } \approx \rbr{ 2 r X_t^{ 2 r - 1 } b \rbr{X_t} + r \rbr{ 2 r - 1 } } \sd t + 2 r X_t^{ 2 r - 1 } b \rbr{X_t} \sd W_t
\end{equation}
and
\begin{equation}
\begin{split}
\ope X_t^{ 2 r } &\le C_1 \int_0^t \rbr{ 2 r \ope \abs{X_t}^{ 2 r - 1 } b \rbr{X_t} + r \rbr{ 2 r - 1 } } \sd t \\
&\le C_1 \int_0^t \rbr{ 2 r \ope \abs{X_t}^{ 2 r - 1 } + 2 r L \ope X_t^{ 2 r } + r \rbr{ 2 r - 1 } } \sd t \\
&\le \int_0^t \rbr{ C_2 \ope X_t^{ 2 r } + C_3 } \sd t \\
\end{split}
\end{equation}
and
\begin{equation}
\ope X_t^{ 2 r } \le \frac{C_3}{C_2} \rbr{ \se^{ C_2 t } - 1 }.
\end{equation}

We have again
\begin{equation}
\begin{split}
\ope X_{ n + 1 }^{ 2 r } &\le \ope X_n^{ 2 r } + 2 C_4 r \ope \abs{X_n}^{ 2 r - 1 } b \rbr{X_n} \Delta t_n \\
&\le \rbr{ 1 + C_5 \Delta t_n } \ope X_n^{ 2 r } + C_6 \Delta t_n
\end{split}
\end{equation}
and this leads to
\begin{equation}
\ope X_n^{ 2 r } \le \frac{C_6}{C_5} \rbr{ \se^{ C_6 T } - 1 }.
\end{equation}

Since $ \ope X_n^{ 2 r } \le C $, and $ \ope \rbr{ \overline{X}_t - X_n }^2 \le C_7 \Delta t $, we deduce $ \overline{X}_t^{ 2 r } \le C' $.
\hfill$\Box$

\end{document}
