%! TeX encoding = UTF-8
%! TeX program = LuaLaTeX

\documentclass[english, nochinese]{pnote}
\usepackage[paper]{pdef}

\DeclareMathOperator\oppr{\mathrm{Pr}}

\title{Answers to Exercises (Lecture 06)}
\author{Zhihan Li, 1600010653}
\date{November 4, 2018}

\begin{document}

\maketitle

\textbf{Problem 1.} \textit{Answer.} Since the transition matrix is
\begin{equation}
P = \msbr{ & 1 & & & & & \\ 1 / N & & \rbr{ N - 1 } / N & & & & \\ & 2 / N & & \rbr{ N - 2 } / N & & & \\ & & 3 / N & & \ddots & & \\ & & & \ddots & & 2 / N & \\ & & & & \rbr{ N - 1 } / N & & 1 / N \\ & & & & & 1 & },
\end{equation}
the equation
\begin{equation}
\pi P = \pi
\end{equation}
gives
\begin{equation}
\begin{cases}
\pi_0 = \pi_1 / N; \\
\pi_1 = N \pi_0 / N + 2 \pi_2 / N; \\
\pi_2 = \rbr{ N - 1 } \pi_1 / N + 3 \pi_3 / N; \\
\vdots \\
\pi_{ N - 1 } = 2 \pi_{ N - 2 } / N + N \pi_N / N; \\
\pi_N = \pi_{ N - 1 } / N.
\end{cases}
\end{equation}
This equation yields
\begin{equation}
\pi_k = \binom{n}{k} \pi_0
\end{equation}
for $ 0 \le k \le n $. Normalization equation
\begin{equation}
\pi 1 = 1
\end{equation}
yields
\begin{equation}
\pi_k = \frac{1}{2^n} \binom{n}{k}.
\end{equation}

\textbf{Problem 2.} \textit{Answer.} Denote the characteristic function of $X_t$ to be $ f_t \rbr{s} $. According to the independence and stability of increments, we have for any $ h_1, h_2, \cdots, h_n > 0 $,
\begin{equation}
f_{ h_1 + h_2 + \cdots + h_n } = f_{h_1} f_{h_2} \cdots f_{h_n}.
\end{equation}
This means
\begin{equation}
f_t \rbr{s} = \rbr{ f_{ t / n } \rbr{s} }^n.
\end{equation}
It follows from the hypothesis of $X_t$ that
\begin{equation}
\begin{split}
f_{ t / n } \rbr{s} &= \rbr{ 1 - \lambda \frac{t}{n} + o \rbr{\frac{1}{n}} } + \rbr{ \lambda \frac{t}{n} + o \rbr{\frac{1}{n}} } \se^{ \si s } + o \rbr{h} \\
&= 1 + \frac{1}{n} \lambda t \rbr{ \se^{ \si s } - 1 } + o \rbr{\frac{1}{n}}.
\end{split}
\end{equation}
Hence, forcing $ n \rightarrow \infty $,
\begin{equation}
f_t \rbr{s} = \exp \rbr{ \lambda t \rbr{ \se^{ \si s } - 1 } },
\end{equation}
which is exactly the characteristic function of $ \mathcal{P} \rbr{ \lambda t } $. Inverse theorem applies and therefore $ X_t \sim \mathcal{P} \rbr{ \lambda t } $.

\textbf{Problem 3.} \textit{Answer.} Denote the column vector of $ h_i \rbr{t} $ and $ f \rbr{i} $ to be $ h \rbr{t} $ and $f$ respectively. Since
\begin{equation}
h_i \rbr{t} = \sum_j p_{ i j } \rbr{t} f \rbr{j},
\end{equation}
we have
\begin{equation}
h \rbr{t} = P \rbr{t} f.
\end{equation}
Differentiating with respective to $h$ yields the equation
\begin{equation}
\dot{h} \rbr{t} = \dot{P} \rbr{t} f = Q P \rbr{t} f = Q h \rbr{t}.
\end{equation}
We may further add the equation
\begin{equation}
h \rbr{0} = f
\end{equation}
to define a initial value problem.

\textbf{Problem 4.} \textit{Answer.} The distribution at $t$, say $X_t$, is the sum of $ t / \tau $ independent samples from $ \mathcal{B} \rbr{ 1, p } $. This means $ X_t \sim \mathcal{B} \rbr{ t / \tau, p } $. When we keep $ p / \tau = \lambda $ when $ \tau \rightarrow 0 $, we obtain that $ \mathcal{B} \rbr{ t / \tau, p } $ converges to $ \mathcal{P} \rbr{ \lambda t } $. In this case, the independence and stability can be passed from discrete settings to the continuous one. Hence, the limiting process is a Poission process intuitively.

In conclusion, the condition we need intuitively is $ p = \lambda \tau $ with $ \tau \rightarrow 0 $.

\textbf{Problem 5.} \textit{Answer.} The equation of $p_0$ is
\begin{equation}
p_0 \rbr{ t + h } = p_0 \rbr{t} \rbr{ 1 - \lambda \rbr{t} h + o \rbr{h} }
\end{equation}
and therefore
\begin{equation}
\dot{p}_0 \rbr{t} = -\lambda \rbr{t} p_0 \rbr{t}.
\end{equation}
Define
\begin{equation}
\Lambda \rbr{t} = \int_0^t \lambda \rbr{s} \sd s,
\end{equation}
and we derive from $ p_0 \rbr{0} = 0 $ that
\begin{equation}
p_0 \rbr{t} = \exp \rbr{ -\Lambda \rbr{t} }.
\end{equation}
Again from
\begin{equation}
p_{ k + 1 } \rbr{ t + h } = p_k \rbr{t} \rbr{ \lambda \rbr{t} h + o \rbr{h} } + p_{ k + 1 } \rbr{t} \rbr{ 1 - \lambda \rbr{t} h + o \rbr{h} } + o \rbr{h}
\end{equation}
or
\begin{equation}
\dot{p}_{ k + 1 } \rbr{t} = - \lambda \rbr{t} \rbr{ p_{ k + 1 } \rbr{t} - p_k \rbr{t} }
\end{equation}
we derive
\begin{equation}
p_k \rbr{t} = \frac{\rbr{ \Lambda \rbr{t} }^k}{ k ! } \exp \rbr{ -\Lambda \rbr{t} }.
\end{equation}
This is equivalent to $ X_t \sim \mathcal{P} \rbr{ -\Lambda \rbr{t} } $. For the waiting time, we consider $ t = 0 $ first. By noticing that
\begin{equation}
\oppr \cbr{ \mu_0 > s } = \oppr \cbr{ X_s = 0 } = \exp \rbr{ -\Lambda \rbr{s} },
\end{equation}
the density function of $\mu_0$ is
\begin{equation}
q_0 \rbr{s} = \lambda \rbr{s} \exp \rbr{ -\Lambda \rbr{s} }.
\end{equation}
Back to arbitrary $t$, it follows from independence of increments that
\begin{equation}
q_t \rbr{s} = \lambda \rbr{s} \exp \rbr{ -\int_t^s \lambda \rbr{u} \sd u } = \lambda \rbr{s} \exp \rbr{ -\Lambda \rbr{s} + \Lambda \rbr{t} }.
\end{equation}

\end{document}
