%! TeX encoding = UTF-8
%! TeX program = LuaLaTeX

\documentclass[english, nochinese]{pnote}
\usepackage[paper, cgu]{pdef}

\newcommand\noteas{\text{\textit{a.s.}}}
\DeclareMathOperator\ope{\mathrm{E}}

\title{Answers to Exercises (Lecture 11)}
\author{Zhihan Li, 1600010653}
\date{November 28, 2018}

\begin{document}

\maketitle

\textbf{Problem 1.} \textit{Answer.} We first observe by $ \ope \xi = 1 / 3 $ and strong law of large number that
\begin{equation}
\frac{S_{\fbr{ N t }}}{N} \rightarrow \frac{1}{3} t, \noteas.
\end{equation}
This means for any $ \alpha > 1 $, $ Z_t^N \rightarrow 0 $ almost surely and for $ \alpha < 1 $, $Z_t^N$ diverges almost surely. As a result, the only choice is $ \alpha = 1 $ and therefore $ Z_t = t / 3 $ is a deterministic process.

\textbf{Problem 2.} \textit{Answer.} (a). Since for fixed $t$
\begin{equation}
W_t \sim \mathcal{N} \rbr{ 0, t },
\end{equation}
we have
\begin{equation}
\ope W_t^4 = 3 t^2.
\end{equation}

(b). We have
\begin{equation}
\begin{split}
&\ptrel{=} \ope \rbr{ W_t - W_s + W_z }^2 \\
&= t + s + z - 2 t \wedge s - 2 z \wedge s + 2 t \wedge z \\
&=
\begin{cases}
t - s + z, & t \le s \le z; \\
t - z + s, & t \le z \le s; \\
-3 s + 3 t + z, & s \le t \le z; \\
-3 s + 3 z + t, & s \le z \le t; \\
z - t + s, & z \le t \le s; \\
z - s + t, & z \le s \le t.
\end{cases}
\end{split}
\end{equation}

\textbf{Problem 3.} \textit{Answer.} We have
\begin{equation}
\begin{split}
&\ptrel{=} \ope \exp \rbr{ -\frac{1}{2} X^{\text{T}} B X } \\
&= \int \frac{1}{ \sqrt{ 2 \spi }^n \sqrt{ \det A } } \exp \rbr{ -\frac{1}{2} X^{\text{T}} A^{-1} X } \exp \rbr{ -\frac{1}{2} X^{\text{T}} B X } \\
&= \frac{\sqrt{ \det C }}{\sqrt{ \det A }} \int \frac{1}{ \sqrt{ 2 \spi }^n \sqrt{ \det C } } \exp \rbr{ -\frac{1}{2} X^{\text{T}} C^{-1} X } \\
&= \frac{1}{\sqrt{ \det A \det C^{-1} }} \\
&= \frac{1}{\sqrt{ \det \rbr{ I + A B } }},
\end{split}
\end{equation}
where $C$ is the positive definite matrix defined as
\begin{equation}
C^{-1} = A^{-1} + B.
\end{equation}

\textbf{Problem 4.} \textit{Proof.} We should only verify the equivalence between 1, 2 and 1$'$ and 2$'$. We assume 1 and 2 first. Since $W_t$ is a Gaussian process, $ \rbr{ W_s, W_{ s + t } } $ follows some Gaussian distribution and so is $ W_{ s + t } - W_s $. Since $ m \rbr{t} = 0 $, it is also of zero expectation. Since
\begin{equation}
\ope \rbr{ W_{ s + t } - W_s }^2 = s + t + s - 2 \rbr{s} = t,
\end{equation}
we deduce
\begin{equation}
W_{ s + t } - W_s \sim \mathcal{N} \rbr{ 0, t }.
\end{equation}
Similar arguments yields $ \rbr{ W_{t_0}, W_{t_1}, \cdots, W_{t_n} } $ follows some Gaussian distribution and so is $ \rbr{ W_{t_0}, W_{t_1} - W_{t_0}, \cdots, W_{t_n} - W_{t_{ n - 1 }} } $. Without loss of generality, we assume $ t_0 = 0 $ and may drop $W_{t_0}$ since it is deterministic zero. As a result, we have for $ i < j $
\begin{equation}
\ope \rbr{ W_{t_{ i + 1 }} - W_{t_i} } \rbr{ W_{t_{ j + 1 }} - W_{t_j} } = t_{ i + 1 } - t_i - t_{ i + 1 } + t_i = 0.
\end{equation}
This combined with Gaussian distribution tells the independence.

We then assume 1$'$ and 2$'$. (We need to further assume $ W_0 = 0 $.) Since $ \rbr{ W_{t_0}, W_{t_1} - W_{t_0}, \cdots, W_{t_n} - W_{t_{ n - 1 }} } $ are independent and each entry follows some Gaussian distribution, the joint distribution is again a Gaussian distribution and so is $ \rbr{ W_{t_0}, W_{t_1}, \cdots, W_{t_n} } $. This means $W_t$ is a Gaussian process. Because
\begin{equation}
W_t \sim \mathcal{N} \rbr{ 0, t }
\end{equation}
for fixed $t$, the mean function $ m \rbr{t} = 0 $. Because for $ s \le t $
\begin{equation}
\ope W_s W_t = \ope W_s^2 + \ope W_s \rbr{ W_t - W_s } = s + 0 = s,
\end{equation}
the covariance function $ K \rbr{ s, t } = s \wedge t $ as desired.
\hfill$\Box$

\end{document}
