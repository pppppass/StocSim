%! TeX encoding = UTF-8
%! TeX program = LuaLaTeX

\documentclass[english, nochinese]{pnote}
\usepackage[Symbolsmallscale]{upgreek}
\usepackage{bm}
\usepackage[paper, cgu]{pdef}

\title{Answers to Exercises (Lecture 14)}
\author{Zhihan Li, 1600010653}
\date{December 26, 2018}

\begin{document}

\maketitle

\textbf{Problem 1.} \textit{Proof.} (2.11). The related dynamic is
\begin{equation}
\sd X^t_i = \rbr{ b_i + \frac{1}{2} \rbr{ \pdl{j} \sigma_{ i k } } \rbr{ \sigma_{ j k } } } \sd t + \sigma_{ i l } \sd W_l^t.
\end{equation}
Hence, the corresponding equation of density $p$ is
\begin{equation}
\begin{split}
p_t &= -\pdl{i} \rbr{ b_i p + \frac{1}{2} \rbr{ \pdl{j} \sigma_{ i k } } \rbr{ \sigma_{ j k } p } } + \frac{1}{2} \pdl{ i j } \rbr{ \sigma_{ i k } \sigma_{ j k } p } \\
&= -\pdl{i} \rbr{ b_i p } + \frac{1}{2} \pdl{i} \rbr{ \sigma_{ i k } \pdl{j} \rbr{ \sigma_{ j k } p } }
\end{split}
\end{equation}
as desired.

(2.12). The related dynamic is
\begin{equation}
\sd X^t_i = \rbr{ b_i + \rbr{ \pdl{j} \sigma_{ i k } } \rbr{ \sigma_{ j k } } } \sd t + \sigma_{ i l } \sd W_l^t.
\end{equation}
The corresponding equation of density $p$ is already
\begin{equation}
p_t = -\pdl{i} \rbr{ b_i p + \rbr{ \pdl{j} \sigma_{ i k } } \rbr{ \sigma_{ j k } p } } + \frac{1}{2} \pdl{ i j } \rbr{ \sigma_{ i k } \sigma_{ j k } p }
\end{equation}
as desired.
\hfill$\Box$

\textbf{Problem 2.} \textit{Answer.} We claim the invariant distribution is Gaussian first. This is because the equation
\begin{equation}
\mathbf{B} \nabla \cdot \rbr{ \bm{x} p } = \frac{1}{2} \upsigma \upsigma^{\text{T}} \Delta p
\end{equation}
is a elliptic equation and the solution is uniquely determined by $ p > 0 $, $ \int p = 1 $. One may directly verify the distribution given in Exercises of Lecture 14 satisfies this equation.

Hence, the detailed balance condition is written as
\begin{equation}
\bm{j} = \mathbf{B} \bm{x} p - \frac{1}{2} \upsigma \upsigma^{\text{T}} \nabla p = 0
\end{equation}
and $ p \sim \mathcal{N} \rbr{ 0, \bm{\Sigma} } $ yields
\begin{equation}
\nabla p = \bm{\Sigma}^{-1} \bm{x} p.
\end{equation}
Combining these all, we reach
\begin{equation}
\mathbf{B} \bm{\Sigma} = \frac{1}{2} \upsigma \upsigma^{\text{T}}.
\end{equation}
\hfill$\Box$

\end{document}
