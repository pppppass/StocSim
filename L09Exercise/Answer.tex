%! TeX encoding = UTF-8
%! TeX program = LuaLaTeX

\documentclass[english, nochinese]{pnote}
\usepackage[paper, cgu]{pdef}

\title{Answers to Exercises (Lecture 09)}
\author{Zhihan Li, 1600010653}
\date{November 7, 2018}

\begin{document}

\maketitle

\textbf{Problem 1.} \textit{Answer.} There are only countable minimizers. We suppose the energy function $V$ has \emph{coercivity}, that is
\begin{equation}
\lim_{ x \rightarrow \infty } V \rbr{x} = +\infty.
\end{equation}
Under this assumption, the minimizers lie in a compact set $K$. If there are infinitely many minimizers, there exists a cluster point, which is still a minimizer and this goes against the isolation. Therefore, there are finitely many minimizers, say $ x_1, \cdots, x_n $. We also denote the minimal value to be $V_0$.

We first show that the limiting distribution concentrates on the minimizers. It is sufficient to prove for any open set $O$ containing all the minimizers, as $ \beta \rightarrow +\infty $,
\begin{equation} \label{Eq:Limit}
\bfrac{ \int_O \exp \rbr{ -\beta V \rbr{x} } \sd x }{ \int_{\mathbb{R}^d} \exp \rbr{ -\beta V \rbr{x} } \sd x } \rightarrow 1.
\end{equation}
Since there are finitely many minimizers, we may find $\epsilon$ such that
\begin{equation}
B := \bigcup_{ i = 1 }^n B \rbr{ x_i, \epsilon } \subseteq O.
\end{equation}
We further assume for $ x \in B $, $ V \rbr{x} < V_0 + \delta $. Noticing that $ \mathbb{R}^d \setminus O $ is a closed set and $V$ is coercive, we may further assume $\epsilon$ is small enough such that
\begin{equation}
V_0 + \delta < \min_{ x \in \mathbb{R}^d \setminus O } V \rbr{x}.
\end{equation}
As a result,
\begin{equation}
\int_O \exp \rbr{ -\beta V \rbr{x} } \sd x > m \rbr{B} \exp \rbr{ -\beta V - \beta \delta }
\end{equation}
and
\begin{equation}
\begin{split}
&\ptrel{<} \bfrac{ \int_{ \mathbb{R}^d \setminus O } \exp \rbr{ -\beta V \rbr{x} } \sd x }{ \int_O \exp \rbr{ -\beta V \rbr{x} } \sd x } \\
&< \int_{ \mathbb{R}^d \setminus O } \exp \rbr{ -\beta \rbr{ V - V_0 - \delta } } \sd x / m \rbr{B} \rightarrow 0
\end{split}
\end{equation}
according to dominated convergence theorem. This proves \eqref{Eq:Limit}.

The limiting distribution on $ x_1, x_2, \cdots, x_n $ is still unknown and we proceed to tackle some special cases. The argument above shows that the limit
\begin{equation}
\int_{\mathbb{R}^d} \exp \rbr{ -\beta V \rbr{x} } \sd x
\end{equation}
is only related to local behaviors of $V$ around $ x_1, x_2, \cdots, x_n $ under the assumption of smoothness and coercivity. Therefore, we may apply Laplace's asymptotic method here. In the following arguments, we assume $ V_0 = 0 $ for brevity. We only tackle the case $ d = 1 $ here.

First we assume the minimizers $ x_1, x_2, \cdots, x_n $ are all second order, i.e. $ V'' \rbr{x_i} \neq 0 $. Directly applying Laplace's conclusion,
\begin{equation}
\int_{ B \rbr{ x_i, \epsilon } } \exp \rbr{ -\beta V \rbr{x} } \sd x \sim \sqrt{ 2 \spi } \frac{1}{\sqrt{ \beta V'' \rbr{x_i} }}.
\end{equation}
where $\epsilon$ is sufficiently small (such that there are no other minimizers in the ball). This means the limiting distribution is actually
\begin{equation}
\sum_{ i = 1 }^n \frac{ 1 / \sqrt{ V'' \rbr{x_i} } }{ \sum_{ j = 1 }^n 1 / \sqrt{ V'' \rbr{x_j} } } \delta_{x_i}.
\end{equation}

If there are $ 2 k $-th order minimizers, then
\begin{equation}
\int_{ B \rbr{ x_i, \epsilon } } \exp \rbr{ -\beta V \rbr{x} } \sd x \sim A_k \frac{1}{ \sqrt[ 2 k ]{ \beta V^{\rbr{ 2 k }} \rbr{x_i} } }
\end{equation}
where
\begin{equation}
A_k = \int_{\mathbb{R}} \exp \rbr{ -\frac{1}{ \rbr{ 2 k } ! } x^{ 2 k } } \sd x.
\end{equation}
This means higher order minimizers will dominate lower order ones. For example, if $ x_1, x_2, \cdots, x_m $ are $ 2 k $-th order minimizers and $ x_{ m + 1 }, \cdots, x_n $ are of lower orders, the limiting distribution is
\begin{equation}
\sum_{ i = 1 }^m \frac{ 1 / \sqrt[ 2 k ]{ V^{\rbr{ 2 k }} \rbr{x_i} } }{ \sum_{ j = 1 }^m 1 / \sqrt[ 2 k ]{ V^{\rbr{ 2 k }} \rbr{x_j} } } \delta_{x_i}.
\end{equation}

What about infinite order minimizers? An example is
\begin{equation} \label{Eq:Path}
V \rbr{x} = x^2 \exp \rbr{-\frac{1}{x^2}}.
\end{equation}
If
\begin{equation}
V \rbr{x} \sim V_i \rbr{ x - x_i }
\end{equation}
where $V_i$ is a smooth symmetric function strictly increasing on $ \srbr{ 0, +\infty } $, we may say intuitively
\begin{equation}
\begin{split}
&\ptrel{\sim} \int_{ B \rbr{ x_i, \epsilon } } \exp \rbr{ -\beta V \rbr{x} } \sd x \\
&\sim 2 \int_0^{\infty} \exp \rbr{ -\beta V_i \rbr{x} } \sd x \\
&= 2 \beta \int_0^{\infty} V_i^{-1} \rbr{t} \se^{ -\beta t } \sd t.
\end{split}
\end{equation}
This means the dominating order is controlled by $V_i^{-1}$. When $V$ is selected to be \eqref{Eq:Path}, we may find the order is greater than
\begin{equation}
\frac{1}{\sqrt{ \ln \beta }}
\end{equation}
and hence again it dominates finite order ones, whose orders are
\begin{equation}
\frac{1}{\sqrt[ 2 k ]{\beta}}.
\end{equation}

\end{document}
