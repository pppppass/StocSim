%! TeX encoding = UTF-8
%! TeX program = LuaLaTeX

\documentclass[english, nochinese]{pnote}
\usepackage{siunitx}
\usepackage[paper, cgu]{pdef}
\usepackage{pgf}

\DeclareMathOperator\opcov{\mathrm{Cov}}
\DeclareMathOperator\ope{\mathrm{E}}
\DeclareMathOperator\opvar{\mathrm{Var}}
\DeclareMathOperator\opkl{\mathrm{KL}}

\title{Answers to Exercises (Lecture 04)}
\author{Zhihan Li, 1600010653}
\date{October 18, 2018}

\begin{document}

\maketitle

\textbf{Problem 1.} \textit{Proof.} We have
\begin{equation}
\begin{split}
&\ptrel{=} \opcov \rbr{ f \rbr{X}, f \rbr{ 1 - X } } \\
&= \ope_X \rbr{ \rbr{ f \rbr{X} - \overline{f} } \rbr{ f \rbr{ 1 - X } - \overline{f} } } \\
&= \int_0^1 \rbr{ f \rbr{x} - \overline{f} } f \rbr{ 1 - x } \sd x
\end{split}
\end{equation}
where
\begin{equation}
\overline{f} = \ope_X f \rbr{X} = \int_0^1 f \rbr{x} \sd x.
\end{equation}
Suppose $ f \rbr{x} - \overline{f} $ is monotonically increasing, and let
\begin{equation}
\xi = \mathop{\arg\min}_x \cbr{ f \rbr{x} - \overline{f} \ge 0 }.
\end{equation}
As a result, $ f \rbr{x} - \overline{f} \le 0 $ for $ x \in \sbr{ 0, \xi } $ and $\ge$ for $ \sbr{ \xi, 1 } $. Moreover,
\begin{equation}
-\int_0^{\xi} \rbr{ f \rbr{x} - \overline{f} } \sd x = \int_{\xi}^1 \rbr{ f \rbr{x} - \overline{f} } \sd x \ge 0.
\end{equation}
As a result,
\begin{equation}
\begin{split}
&\ptrel{=} -\int_0^{\xi} \rbr{ f \rbr{x} - \overline{f} } f \rbr{ 1 - x } \sd x \\
&\ge -\int_0^{\xi} \rbr{ f \rbr{x} - \overline{f} } \sd x f \rbr{ 1 - \xi } \\
&= \int_{\xi}^1 \rbr{ f \rbr{x} - \overline{f} } \sd x f \rbr{ 1 - \xi } \\
&\ge \int_{\xi}^1 \rbr{ f \rbr{x} - \overline{f} } f \rbr{ 1 - x } \sd x
\end{split}
\end{equation}
and this yields
\begin{equation}
\int_0^1 \rbr{ f \rbr{x} - \overline{f} } f \rbr{ 1 - x } \sd x \le 0
\end{equation}
as desired.
\hfill$\Box$

\textbf{Problem 2.} \textit{Proof.} We have
\begin{equation}
\begin{split}
&\ptrel{=} \opvar_{\mathbf{X}^{\rbr{2}}} \ope_{\mathbf{X}^{\rbr{1}}} f \rbr{\mathbf{X}} + \ope_{\mathbf{X}^{\rbr{2}}} \opvar_{\mathbf{X}^{\rbr{1}}} f \rbr{\mathbf{X}} \\
&= \ope_{\mathbf{X}^{\rbr{2}}} \rbr{ \ope_{\mathbf{X}^{\rbr{1}}} f \rbr{\mathbf{X}} }^2 - \rbr{ \ope_{\mathbf{X}^{\rbr{2}}} \ope_{\mathbf{X}^{\rbr{1}}} f \rbr{X} }^2 \\
&+ \ope_{\mathbf{X}^{\rbr{2}}} \ope_{\mathbf{X}^{\rbr{1}}} f \rbr{\mathbf{X}}^2 - \ope_{\mathbf{X}^{\rbr{2}}} \rbr{ \ope_{\mathbf{X}^{\rbr{1}}} f \rbr{\mathbf{X}} }^2 \\
&= \ope_{\mathbf{X}} f \rbr{\mathbf{X}}^2 - \rbr{ \ope_{\mathbf{X}} f \rbr{\mathbf{X}} }^2 \\
&= \opvar_{\mathbf{X}} f \rbr{\mathbf{X}}.
\end{split}
\end{equation}
\hfill$\Box$.

\textbf{Problem 3.} \textit{Proof.} Since $ \ln x $ is concave and
\begin{equation} \label{Eq:Norm}
\int f = \int g = 1,
\end{equation}
\begin{equation}
\opkl \rbr{ f \mnorm g } = \int f \ln \frac{f}{g} = -\int f \ln \frac{g}{f} \ge \ln \int f \frac{g}{f} = 0.
\end{equation}
The equality is attained iff $ g / f $ is a constant \textit{a.e.}, combing which and \eqref{Eq:Norm} it follows
\begin{equation}
f = g \mathrel{a.e.}.
\end{equation}
\hfill$\Box$

\end{document}
